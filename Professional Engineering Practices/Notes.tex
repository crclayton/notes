\documentclass{article}
\usepackage{array, booktabs, graphicx, apacite, setspace}
\usepackage[titletoc,toc,title]{appendix}
\usepackage[toc,section=section]{glossaries}
\usepackage{tocloft}
\usepackage{titlesec}
\usepackage{fancyvrb}
\usepackage{float} % for strict figure placement  with option [H]
\usepackage[bottom]{footmisc} % to glue footnote to bottom of page
\usepackage{alltt} % for bold typewriter
\usepackage{amsthm} % for proof environment
\usepackage{ragged2e} % to undo \centering

% set section indentation
\setcounter{secnumdepth}{4}


% add space between paragraphs
\setlength{\parskip}{\baselineskip}


% format \paragraph{example} as a subsubsubsection
\titleformat{\paragraph}
{\normalfont\normalsize\bfseries}{\theparagraph}{1em}{}
\titlespacing*{\paragraph}
{0pt}{3.25ex plus 1ex minus .2ex}{1.5ex plus .2ex}


% automatically convert "" to ``''
\usepackage [autostyle, english = american]{csquotes}
\MakeOuterQuote{"}


% define list of equations
\newcommand{\listequationsname}{\Large{List of Equations}}
\newlistof{myequations}{equ}{\listequationsname}
\newcommand{\myequations}[1]{
   \addcontentsline{equ}{myequations}{\protect								       \numberline{\theequation}#1
   }
}
\setlength{\cftmyequationsnumwidth}{2.3em}
\setlength{\cftmyequationsindent}{1.5em}


% format appendix numbering
\renewcommand\appendix{\par
  \setcounter{section}{0}
  \setcounter{subsection}{0}
  \setcounter{figure}{0}
  \setcounter{table}{0}
  \renewcommand\thesection{Appendix \Alph{section}}
  \renewcommand\thefigure{\Alph{section}\arabic{figure}}
  \renewcommand\thetable{\Alph{section}\arabic{table}}
}


% -------- GLOSSARY ENTRIES -------- %
\makenoidxglossaries

\newglossaryentry{random experiment}{
	name={Random experiment},
	description={Definition of the word}
}


% renames "Contents" to "Table of Contents"
\renewcommand\contentsname{Table of Contents}

\newcommand{\eqn}[1]{\myequations{#1} \centering  \small \textit{#1} \normalsize \justify  }

\newcommand{\var}{\sigma^2}

\newcommand{\n}[1]{\overline{#1}}


\begin{document}

% -------------------------------------- %
% ------------ FRONT MATTER ------------ %
% -------------------------------------- %

% -------- TITLE PAGE -------- %

\title{\huge APSC 450 \\ \Large \medskip Professional Engineering Practices}
\author{Charles Clayton}
\date{\today}
\maketitle
`
\thispagestyle{empty}

\pagenumbering{roman}
\setcounter{page}{0}


% -------- TABLE OF CONTENTS/LISTS -------- %

\singlespacing			\pagebreak
\tableofcontents		\pagebreak

\listoffigures		
\listoftables
\listofmyequations
\pagebreak


% -------- GLOSSARY -------- %

\printnoidxglossaries	\pagebreak


% ----------------------------- %
% -------- MAIN MATTER -------- %
% ----------------------------- %

\pagenumbering{arabic}


There is more to engineering than technology -- there are elements of law, business, safety, psychology, philosophy, and more that professional engineers should be aware of.

\section{APEGBC}

\subsection{Becoming an EIT}

Engineering in BC is self-regulated by APEGBC under the \textit{Engineers and Geoscientists Act}. In order to undertake engineering projects in BC you must be licensed by APEGBC.

First you must be a student APEGBC member and complete an undergraduate degree in engineering. 

To apply to be an EIT, you must pay the \$472.50 application fee\footnote{Waived if you apply within 12 months of graduation}, complete the online application, provide ID signed by a P.Eng, and send all transcripts from post-secondary institutions to APEGBC. Thereafter there are \$232.05 yearly membership fees. The application  takes about two weeks to process. You do not need to be currently employed to be an EIT.

\subsection{Becoming a P.Eng}
Then you must complete four-years of relevant work experience as an EIT under the supervision of a P.Eng, must have had at least one-year of work experience in Canada, and complete a Law and Ethics Seminar and complete the Professional Practice Exam (PPE) which is 3-hours long including a 1-hour long essay.

You will also have to report your work experience using APEGBC's online competency experience program. Then you can be a P. Eng and practice engineering in BC. You will need a minimum of four references, one of which will share the same discipline of practice. These can include clients and consultants -- references must be knowledgeable about your work.


\subsection{Professionalism}

The APEGBC Code of Ethics states: \begin{quote}Members and
licensees shall act at all times with fairness, courtesy
and good faith to their associates, employers,
employees and clients, and with fidelity to the public
needs. They shall uphold the values of truth, honesty
and trustworthiness and safeguard human life and
welfare and the environment.
\end{quote}

Professionalism can matter with respect to competency, conflict of interest, confidentiality, use of company documentation, respect for privacy, environmental responsibility, cultural sensitivity, equity, discrimination, harassment, occupational health and safety, etc.

Engineers are responsible to demonstrate professionalism and adhere to the APEGBC Code of Ethics.

\subsubsection{Protection of the Public}

APEGBC Code of Ethics states:  \begin{quote}Hold paramount the safety,
health and welfare of the public, the protection of the
environment and promote health and safety within the
workplace
\end{quote}


\subsubsection{Serving the public interest}

Reduce negative societal, economic, and environmental externalities, even when these imply reduced private profits.

\subsection{Self-Regulation}

Rather than be regulated by government, engineers in BC are self-regulated, which means that the regulation body is comprised of the members themselves. APEGBC is given the powers to regulation through the Engineers \& Geoscientists Act. 

Self-regulation may be withdrawn and government regulation introduced, as in the case of the British Columbia College of Teachers or Engineering in Quebec.

APEGBC and other provincial assocations are members of Engineers Canada, which includes the Canadian Engineering Accreditation Board and the Canadian Engineering Qualifications Board, which are responsible for national standards of engineering programs and qualifications.

\subsection{Governance}

APEGBC is governed by a council of 13 elected members, and 4 members appointed by Lieutenant Governor. The council includes a President and VP with one-year terms.


\section{Legal System and Torts}

\subsection{Common Law}

All of Canada other than Quebec operates under something called Common Law. Quebec operates under Civil Law. 

Common law is not written down as a body of rules (legislation). Rather it is judge-made law that is developed case-by-case. It was inherited from the British Empire, and common law is the first major source of law in Canada.


\subsection{Legislation}

Canada negotiated independence from the British Empire in 1867 and wrote its constitution. The constitution gives Canada the power to create/legislate its own laws to supplement or replace common law. Legislation  consists of the constitution, acts, and statutes, and is the second major source  of law in Canada.

In 1982, Canada enacted the Charter of Rights and Freedoms, which gave some power back to the courts.

Some legislation that effects engineering includes, Health and Safety (Workers Compensation Act), intellectual property rights, construction regulations/codes/bylaws, environmental protection, business practices/consumer protection.

Legislation is divided between Federal Jurisdiction and Provincial Jurisdiction. All criminal law, that relating to imprisoning people, is under federal jurisdiction.

Legislation starts with a bill proposed by a minister or an MP, and then must be passed by the house of commons and the senate. Then it must receive royal assent and become a statute.

\subsubsection{Regulations}

Regulations are a subset of legislation that can be enacted without parliament by people/committees given the authority under an Act. Regulations are usually related to technical areas where parliament does not have requisite knowledge. These can be enacted and modified more quickly.

You can find and read legislation on \texttt{bclaws.ca} or \texttt{canlii.org}.

\subsection{Terminology}

The \textbf{plaintiff}, $\pi$, is the person  who makes a claim in a lawsuit. The \textbf{defendant}, $\Delta$, is the person being sued by the plaintiff.

An \textbf{appeal} can be taken when they lost a lawsuit and want the case to be considered by a higher level of court. The \textbf{appellant} is the person who makes the appeal. The \textbf{respondent} is the person who defends the appeal.

A course proceeding/lawsuit is called an \textbf{action}. \textbf{Damages} are the amount of money awarded by a court to make up for a loss. \textbf{Personal property} is almost anything that has value and is not real estate.

A \textbf{tort} is a wrongdoing that involves a breach of civil duty owed to another party. Basically, a wrong between people. A person will take another person to a court and the person who has been wronged can obtain compensation.

A \textbf{crime} is a wrongdoing that involves a breach of duty prescribed by the state. This is a wrong between the state and a person, and the state will take the person to court to try to send them to jail. If someone commits a crime against you, you don't do anything. It's up to the state/police.

Somethings, like punching someone, can be both. The tort is battery, for which the victim can get paid damages by the puncher. The crime is assault, for which the puncher can be punished by the state.

\subsection{Torts}

A tort has three elements: \begin{enumerate}
\item Breach of established common law duty
\item Intent\footnote{Usually, negligence isn't exactly intentional}
\item Harm/losses/damage
\end{enumerate}

The point of a tort is to restore the plaintiff to the way they were before the tort happened by way of compensation. The plaintiff cannot get more out of the law suit (damages) than they had before. In especially bad cases, the court may award \textit{aggravated} or \textit{punitive} damages. If there is no harm, then you will be awarded no damages.

If the damages are less than \$25k, then you will go to small-claims provincial court. You will pay our own legal fees if you win/lose.

With torts, the burden of proof is on the plaintiff, who must prove "on a balance of probabilities" ($> 50\%$ likelihood) that they were wronged. Note that criminal law has the much higher of "beyond a reasonable doubt". In tort law, the plaintiff can make their case with circumstantial law -- things that would make people  \textit{think} that the defendant is guilty.

The focus of a tort is on the plaintiff, whereas the focus of a crime is on the offender. 


\subsubsection{Punitive Damages}


Generally, damages are intended to compensate the victim of a tort (negligence). Courts can and do award punitive damages, which are intended to punish the defendant, rather than compensate the plaintiff.

Courts impose punitive damages (also called aggravated damages) in cases where the court finds that a defendant's conduct has been "high-handed, malicious or reprehensible". Punitive damages skew the decision tree.

Problem: since judgments are retrospective, there is no way to know in advance what the punitive judgment would amount to.




\subsection{Court Structure}

Provincial courts and federal courts have different jurisdictions as defined in the constitution. 

Appeals courts exist to reconsider cases to determine if to pass them on to the provincial or the federal supreme courts.

BC Provincial court has jurisdiction for criminal law, family law, small claims, traffic infractions, and bylaw matters. The BC Supreme court, by definition, covers everything not allocated to provincial court. 

\section{Ethics}

Ethics deal with choices made by people related to right/wrong, good/evil, justice. They refer to societal and cultural customs of behaviour, whereas morals concern individual standards of right and wrong. Ethics don't have any proof of being correct, but rather are based upon agreed standards of conduct in a society.

Many laws are based on ethical principles, concerning crimes and equity, whereas other laws such at tax laws are simply practical. 

The APEGBC code of ethics is as follows (simplification in bold).

\begin{enumerate}
\item \textbf{Public interest}

Hold paramount the safety, health and welfare of the
public, protection of the environment and health and
safety within the workplace. 
\item \textbf{Know your limits}

Accept responsibility for professional assignments only
when qualified by training or experience.
\item \textbf{Don't fake it}

Provide an opinion on a professional subject only when
based on adequate knowledge and honest conviction.

\item \textbf{Conflict of interest
}

Act as faithful agents of clients or employers, maintain
confidentiality and avoid conflicts of interest. Where
conflicts arise, fully disclose the circumstances without
delay to the employer or client

\item \textbf{Respect your value}

Uphold the principle of appropriate and adequate
compensation for the performance of engineering work.

\item \textbf{Lifelong learning}

 Keep informed in order to maintain competence. Strive to advance the body of knowledge. Provide opportunities
for professional development of associates.

\item \textbf{Do unto others...}

Act with fairness, courtesy and good faith towards
clients, colleagues and others. Give credit where due.
Accept and give honest and fair professional comment.

\item \textbf{Stand your ground}

Present clearly to employers and clients the possible
consequences if professional decisions or judgments are
overruled or disregarded.

\item \textbf{Be brave (whistleblowing)}

 Report to the Association or other appropriate agencies
any hazardous, illegal or unethical professional decisions
or practices by engineers or others.

\item \textbf{Spread the word}

 Extend public knowledge and appreciation of
engineering. Protect it from misrepresentation and
misunderstanding.

\end{enumerate}

Four practical philosophical theories:

\begin{enumerate}
\item Aristotle -- Actions are right if they support good character traits (virtues). Virtue ethics consider an action correct if the action is honest, demonstrates loyalty to community or employer, is believed to be responsible, and avoids extremes (is a compromise).

\item Locke -- Each person has rights which need to be respected. These include freedoms of conscience, religion, thought, expression, and so on; democratic rights; legal rights, such as life, liberty, and security; and equity rights such as rights against discrimination.

\item Kant -- Every person must discharge duties. Duties are obligations and rules of conduct.

\item Mill - Utilitarianism is governed by the maximum benefit for the greatest number of people 

\end{enumerate}

\section{Negligence}

Remember that most torts required \textit{intent}. However, negligence only requires the following:

\begin{enumerate}
\item The defendant ($\Delta$) owes the plaintiff ($\pi$) a duty of care -- Was there forseeable damage?
\item $\Delta$ did not meet the requisite standard of care 
\item The $\Delta$'s breach of the standard caused harm or loss to the $\pi$ -- "But for test?"
\end{enumerate}



\subsection{Duty of care}

There is a duty of care when it is "reasonably foreseeable" that harm could occur. Once a duty of care has been established for a given situation by a previous case, by common law it sets a precedent for all future situations of a similar nature. Engineers owe a duty of care to clients, customers, and the public.

Note that legal activity does not immediately imply failure to meet a duty of care. If it's a sunny day and you have lots of visibility on a clear rode, you can be speeding and still meeting the standard of care. If it's an icy snow storm and you're driving the speed limit, you may still not be meeting the standard of care.

The standard of care for engineers is that:

\begin{quote}
A practising engineer or
geoscientist must exercise the same degree of
skill and possess the same level of knowledge
that is generally expected of other members of
the profession with the same training and
experience.
\end{quote}

So junior engineers are held to a different standard than senior engineers. Junior engineers are not expected to know everything, but they are expected to ask for help from senior engineers in situations where they have insufficient knowledge.

\subsection{Causation}

In order to be negligent, there does not need to direct causation to the harm. To test for causation, you can use the \textit{but for} test -- "would the harm have occurred but for the actions of the defendant?"

\subsection{Damages}

Damages are designed co compensate for the loss suffered. The damages can be reduced if there is \textbf{contributory negligence}, which means the plaintiff is somewhat at fault for the incident. These can include failure to mitigate damages.

\subsection{Liability}

Negligence falls under strict liability, or liability without intent -- regardless of precautions. Strict liability is usually not applied in Canadian courts. It is more often used in American courts relating to transporting dangerous chemicals and damage to the environment.

\textbf{Product liability} is where products have caused damage. Breach implied contractual undertaking that the product is not defective. Also, manufacturers must warn consumers if there are possible side-effects or risks.

\textbf{Vicarious liability} is that an employer is often liable for the actions  of their employee in the line of work. Though an employer is usually not liable for \textit{contractors}. An employer may still sue the engineer to recover any losses. Engineers may get insurance to cover this.

\textbf{Third-party liability} is where liability can extend to people with whom the defendant had no contact or contract with, but that were effected by the negligence.

To be \textbf{jointly liable} means that each party are each liable up to the full amount, but if one party pays then the other party cannot be sued to pay as well. To be \textbf{severally liable} is where both parties are liable only for their respective obligations.  

\textbf{Joint and several} liability means that it us up to the defendants to sort out their respective proportions of the damages, but each party is liable up to the full amount of the damages. If  one defendant doesn't have enough money to pay their share of the damages, the other defendants have to make it up.

\subsection{Limitation Period}

The limitation act says that a claim cannot be brought to court for most things further than 2 years from the day it was discovered in BC. There are some exceptions, for instance if the wrongdoer acknowledges the wrongdoing then the 2 year period resets.


\section{Conflict of Interest}

A situation in which a professional has a special interest that influences the objective exercise of professional duties.

These special interests are anything that render a professional's judgment less reliable than it would otherwise be. These include: \begin{enumerate}
\item Self interest, such as \begin{enumerate}
\item career goals
\item financial gain
\item reputation
\end{enumerate}
\item financial relationships with family or friends
\item loyalty to a group
\item emotional connection
\end{enumerate}

As a professional, engineers have a duty to maintain confidentiality; remain impartial in decisions, evaluations, and design choices; fiduciary duties -- the highest standard of care, those whereby the engineer has been given the utmost trust by an individual.

Engineers are trusted to remain reliable and make unbiased decisions. Failure to do so results in damage to reputation, career, employer, and the profession, and may result in criminal or civil charges.

The APEGBC Code of Ethics states than professional engineers shall... \begin{quote}
Act	 as	 faithful	 agents	 of	 their	 clients	 or	 employers,	
maintain	confiden2ality	and	avoid	conflicts	of	interest	but,	
where	such	conflict	arises,	fully	disclose	the	circumstances	
without	delay	to	the	employer	or	client.
\end{quote}

So if a conflict of interest arises, first try to avoid it. If you cannot, disclose it to your employer and client. Finally, recuse it (disqualify yourself from related decisions).

\subsection{Whistleblowing}

Whistleblowing is raising a concern about wrongdoing within an organization -- such as misconduct, or dishonest/illegal activities. The whisteblower is fulfilling their duty to safeguard the public and exercising the right to disclose wrongdoing.

Internal whistleblowing is where the employee goes over the head of their immediate supervisor. External whisteblowing is where the employee reports the wrongdoing to the media, third-party experts, or law-enforcement.

In order for whistleblowing to be justified: \begin{enumerate}
\item The matter must be important -- that is, failure to act will harm the public or environment
\item The whistleblower must have proof or first-hand knowledge and relevant expertise
\item Other actions within the power of the employee have been exhausted
\end{enumerate}

\section{Persuasive Communication}

Know your audience. Know your purpose. Be brief. Use storytelling.

Hollywood: 

Act 1: Set up characters, setting, challenge, or conflict

Act 2: Develop the action

- How your why points are true

Act 3: Provide resolution to conflict

\section{Employment Law}

\subsection{Sources of Employment Law}

When working for most business you will be subject to provincial law. For federally regulated industries, like banks and airlines, you will be subject to federal law.

\subsection{Minimum standards}

\begin{enumerate} \itemsep-0.25em
\item Minimum wage (\$10.85/hr in BC in 2016)
\item Overtime pay (1.5x for $>$8hrs, 2x for $>12$hrs)
\item Vacation (2 week for 1-3 years, 3 weeks for more)
\item Termination and notice (1 week/yr of employment)
\item Statutory holidays
\end{enumerate}

But none of these minimum standards apply to engineers or EITs, or other professionals like lawyers or doctors or researchers or high-tech (software development etc). The idea being that professionals are capable of managing their own time. The job is generally accepted as hard, and your employer can work you hard.

Independent contractors are not entitled to minimum standards, tax deductions, EI benefits, etc.

\subsection{Occupational Health and Safety}

WorkSafeBC is aimed at protecting workers from health and safety risks created by business activities. Workers have three fundamental rights: \begin{enumerate} \itemsep-0.25em
\item Right to participate 
\item Right to know
\item Right to refuse unsafe work
\end{enumerate}


\subsection{Human Rights}

All provinces have human rights legislation. These protect people in many situations from being discriminated against on prohibited grounds, such as race, sex, age, etc.

\subsection{Ending the employment relationship}

Employees can be terminated with cause, \textit{just cause}, or can be terminated without cause. If terminated without cause, employer has to give notice and severance.


Where the employee resigns, the employee must give reasonable notice -- usually two weeks but depends on company.


\subsubsection{Just Cause}

This is rare, but you are not entitled to severance or notice. There must be serious misconduct and a break down of the employee-employer relationship. Generally requires formal warnings first, otherwise not justified.

\subsubsection{Wrongful Dismissal Damages}

In the case of being terminated without cause/wrongful dismissal, the employer must give notice or pay in lieu. Reasonably, this is about 1-month pay per year of service. The minimum in the employment standards act is a week, but typically you can get more than that especially if you work for them for longer.

If you're getting benefits, if you're getting a car allowance, anything that you've gotten during you're employment, you can be compensated accordingly.

\subsubsection{Constructive  Dismissal}

This is if the employer totally violates the employment contract or the employer made it impossible for the employee to work there through bad behaviour. The employee can quit without notice and claim constructive dismissal. 

Employee must notify the employer of objections. Then the employee can sue the employer for wrongful dismissal and obtain compensation in lieu of notice.

\subsubsection{Duty to Mitigate}

In any case where an employee is terminated, the employee has a duty to mitigate their losses. They have to make reasonable efforts to get another job.

\subsubsection{Non-competition and Non-solicitation agreements}

Restrictions on your actions continue to end after employment ends. You cannot compete for clients/customers of your former employer. These agreements are often unenforceable because courts do not want to restrict people from working. The courts are more favourable to the employee than the employer, so the agreements will only be upheld by courts if they are reasonable in their scope, confined geographic regions and only for a period of time.

\section{10 Rules of Business}

\begin{enumerate}
\item \textbf{Leadership is confidence and vulnerability} \begin{itemize}
\item Ask for/accept help
\item Be humble
\item Understand your impact on and perception by other
\end{itemize}
\item \textbf{Know yourself} \begin{itemize}
\item Find out your own values/purpose
\item Use your values to help you make decisions
\end{itemize}
\item \textbf{Get the basics right} \begin{itemize}
\item Be on time
\item Write thank-you notes
\item Be polite
\end{itemize}
\item \textbf{Learn through failure} \begin{itemize}
\item If you never fail then you're not taking enough risks
\item Taking \textit{personal} risk as opposed to company/technical risks differentiates leaders
\end{itemize}
\item \textbf{Keep energy banks full} \begin{itemize}
\item Eat/sleep/exercise
\end{itemize}
\item \textbf{Read} \begin{itemize}
\item If someone recommends a book, read it
\end{itemize}
\item \textbf{Bosses don't decide who gets promoted, peers do} \begin{itemize}
\item You'll only lead if people want you to
\item Be kind to people 
\item Support other peoples' success
\item Have courage/hire smarter people than you
\end{itemize}
\item \textbf{Life balance} \begin{itemize}
\item Professionals get professional help  whether childcare, housework, finance advice, health care counseling, coaching -- this is a differentiator
\end{itemize}
\item \textbf{Parallel career} \begin{itemize}
\item Find a passion and do it
\item Use it to develop leadership skills
\end{itemize}
\item \textbf{Keep options open} \begin{itemize}
\item Option value is least understood and most powerful resource 
\end{itemize}
\end{enumerate}

\section{Case Studies in Engineering/Ethical Failures}

\subsection{Kansas City Hyatt Walkway Collapse}

The 40 story Hyatt Regency in Kansas City was constructed in 1978-1980. The atrium lobby had several suspended walkways at the second, third and fourth floor levels. The fourth floor walkway was suspended directly over the second floor walkway; the third floor walkway was offset. 

1600 people were in the lobby on July 17, 1981 to participate in and watch a "tea dance". Many stood on the walkways to watch the dance on the floor of the lobby.

 The fourth floor walkway collapsed onto the second floor walkway, which in turn collapsed onto the floor.  114 lives were lost.

A contractor suggested a change to ease fabrication  by using two hole and nut pairs in a crossbeam instead of one.

The result was that the load on the nuts holding the fourth floor walkway was twice as large and the walkway collapsed.

The contractor claimed to have phoned the design engineer for approval, but the engineer denied ever receiving the call. There was no documentation of the change other than shop drawings.

The design engineer delegated a technician was tasked with reviewing the drawings, but the technician performed to calculations of the loads on the connection. 

The design engineer testified that it was common industry practice for structural engineers to leave the design of steel-to-steel connections to fabricators, then performed "spot checks" on portions of the shop drawings and approved them.

As a result, several engineers lost their licenses and firms went bankrupt.

\subsection{DC-10 Cargo Door}

The DC-10 was McDonell Douglas's answer to Boeing's 747. Developed in the late 1960s, the DC-10 began flying in 1970 and entered commercial service in 1971.

On March 3, 1974, Turkish Airlines flight 981, a DC-10 flying out of Paris experienced an explosive decompression 10 minutes after takeoff when a rear cargo door burst open.

The resulting differential in pressure between the passenger and cargo compartments collapsed the floor, severing the hydraulic control lines that ran within the floor. The aircraft crashed and 346 people were killed.


The problem was that although the passenger doors opened inwards, "plug-type", the cargo doors had the latches face the opposite direction and the doors opened outward. If the doors weren't completely closed, then pressure could force the latches open.

This was because they used electric actuators instead of hydraulic actuators. Hydraulic actuators apply constant pressure, but the electric ones turn off after use and ratchets hold the latches in place, but these ratchets failed.

The design engineer recommended hydraulic actuators for this reason, but was overruled because of the lower weight and maintenance ease of the electric actuator.

DC-10 cargo doors had failed prior to this crash – both doors blew off in 1970 during a ground test and the American Airlines flight had experienced a similar explosive decompression over Windsor, Ontario in 1972. The American Airlines flight landed safely, however. The incident was blamed on a mechanic who didn't properly close the door, and some small modifications were made to the door including an indicator showing if it was properly closed. Other modifications were failed to be implemented due to failure of the FAA to ground the aircraft and convey the urgency of the modifications.

An engineer called these "bandaid fixes" and predicted that the safety of the doors would degrade and they would come open, resulting in the loss of the airplane. The engineer's bosses did not act on the memo because they thought it would harm the relationship between companies.

\subsection{Ford Pinto}

Crash tests showed the Pinto wouldn't withstand the standards proposed by the National Highway Traffic Saftey Administration -- stating that cars needed to be able to survive a 20 mph crash without leaking fuel.

A rubber-bladder modification estimated to cost \$11 per vehicle would prevent the leaking, but would push back the introduction of the Pinto. Ford decided to bring it to market without the modification.

There was an internal document with calculations weighing the options. At the time the law valued the a like at \$200k and serious burn injuries at \$67k. Ford estimated selling 11M Pintos and anticipated 180 deaths and 180 burn injuries. 

As a result, Ford calculated the modifications would cost \$121M whereas the lawsuits would cost \$40M and didn't do the modifications.

\subsection{Economic Theory of Negligence (Learned Hand Rule)}

This states the the defendant's duty to protect against injuries is a function of three variables: \begin{enumerate}
\item The probability of the accident, $P$
\item The gravity of the resulting loss/injury, $L$
\item The burned of the adequate precautions, $B$
\end{enumerate} Whereby the expected loss must be greater than the amount of burden to create a duty of care: $$PL > B$$ 

\section{10 Things I Wish I Learned in Engineering School}

\begin{enumerate}
\item \textbf{Overconfidence bias} -- "I know what I'm doing", 90\% of drivers rate themselves as above average
\item \textbf{Confirmation bias} -- "It's worked this long"
\item \textbf{Sunk cost fallacy} -- "I might as well ride it out"
\item \textbf{Rare events happen}, and they have massive impacts
\item \textbf{Most measurements are imprecise}  -- experts are useless, no better than average public
\item \textbf{Gambler's fallacy} -- "hot hand"
\item \textbf{Loss aversion} -- people really hate losing stuff, even if your expected value of the risk outweighs the expected value of the non-risk
\item \textbf{Don't buy from beautiful people} 
\item \textbf{Be wary of the narrative}
\item \textbf{Don't trust memory}  unless its made of silicon
\item \textbf{Beware averages} -- they hide important information
\item \textbf{Seek out things with options} -- options can have asymmetric payoffs
\end{enumerate}

\section{Contracts}

Contracts are a legally enforceable promise between parties. It's a common-law recognition that people should make good on their promises.

Form and content are largely unconstrained -- can be oral or written, with 2 or more parties, unilateral or bilateral, but have four requirements to be enforced: \begin{enumerate}
\item \textbf{Intention to create legal obligations} -- can't be any duress or incapacity
\item \textbf{Offer} -- promise that contains sufficient details so everyone knows what's required of them\footnote{Different from \textbf{invitation to treatt}, which is an advertisement or willingness to negotiate, if someone accepts it that doesn't mean your contractually obligated to go through with it}
\item \textbf{Acceptance and communication of acceptance} -- can't force a contract on someone, \textit{a counter-offer rescinds the offer}
\item \textbf{Consideration} -- something must be given or done in exchange for the promise, gratuitous promises are unenforcable. Consideration can be nominal though.
\end{enumerate}

Freedom of contract is highly regarded in courts, so courts often will try to enforce contracts. It will not enforce contracts wherein it is for something illegal \textbf{illegal}, or where one or more parties did not have \textbf{capacity}, or where there was any \textbf{duress/undue influence}, or any \textbf{misrepresentation}.

\subsection{Remedies}

These are when one party fails to perform obligations, then the contract is breached. You may then sue for \textbf{breach of contract} and the award will be damages. Damages will put you in the position you would have been in had the contract been performed.

If the contract is something unique, the court can require the parties to carry out their obligations, called \textbf{specific performance}.


\subsection{Limitation clauses}

You can limit your liability with a limitation or exclusion cause, but \textbf{you have to bring attention to it} -- use bold type, require signing/initializing clause, etc.

In class we discussed a case of a home inspector getting penalized \$200k because he did not bring enough attention to the limitation of liability clause. 

As engineers, you still may be potentially liable for negligence.

\subsection{Tender Contracts}

Engineers may deal with contractors in a tendering process. The engineer will oversee bids for a contract. Each bid by a contract forms a \textbf{contract A} with the owner -- this is not an invitation to treat, so there can be liability at the bidding stage, such as changing the terms for certain bidders but not all. 

Once the bid is accepted, \textbf{contract B} is formed.

\section{Business Structures}

People complicate their business beyond sole proprietorship for reasons including:
\begin{enumerate}
\item protecting themselves and others from liability
\item gaining tax advantages, providing a means to grow through investment
\item giving the appearance of professionalism
\item providing posterity for the business to survive 

\end{enumerate}

\subsection{Sole Proprietorship}

This is the simplest form of business -- there is no separate legal entity, you are the business. It is easy and cheap to form/dissolve, and the business also dies when you die. 

This is a tax advantage because you can deduct business expenses -- but there is no tax deferral and no special tax rate.

It is appropriate for mom and pop shops, tradespeople, hobbyists, consultants, and people who like to start businesses that go nowhere. 

The main risk here is that there is \textbf{unlimited liability}, you are personally on the hook if your business is sued.

You need to reserve a name online which costs about \$30, then register the business online costing another \$40 dollars.If you want to serve customers on your premises, it'll cost about another \$200 to get a city business license. You also have to register for GST if your revenue is $>$\$30k/yr.

For your business name, you have to have a $distinctive$ element, such as your name or whatever, and then a $descriptive$ element which is what the company does. You can then have a suffix, such as "co." or "Ltd".

\subsection{Partnership}

These are where two or more people carry on a business with a view of profit, sharing profits, expenses, liability, and administration. In BC, these can be \begin{enumerate}\item General Partnerships -- debts incurred by one partner are shared by the others
\item Limited Partnerships or Limited Liability Partnerships -- limits liability of partners to their contribution
\end{enumerate}

Partnership registration and fees the same as the proprietorship. These are harder to set-up and more money is needed because you'll need a lawyer for the partnership agreement.

There are tax advantages, especially to Limited Partnerships.

\subsection{Corporations}

Corporations are separate legal entities. They can sue and be sued, obtain insurance, own property, enter into contracts, purchase products, and live forever.

The point of the separate legal entity is that liability is shifted away from you and your partners to the corporate entity, protecting the owners from debts and liabilities. You can't use the corporation completely to shield yourself from liability, courts will "pierce the corporate veil" where appropriate.


\textbf{Shareholders} are the owners of the company, making ultimate decisions. The \textbf{board of directors} are appointed by the shareholders to run the company. The \textbf{executives} carry out day-to-day management as directed by the board. 

\textbf{Shares} represent fractional ownership of a corporation, and owners are entitled to \textbf{dividends}, capital on windup, and to choose directions/vote on major decisions.

Directors owe the company a \textbf{fiduciary duty}, as well as duties of care, and confidentiality -- so they can be liable for conflicts of interest, taking corporate opportunities, or incompetence -- unhappy shareholders can effectively \textit{sue} directors (this happens frequently in hostile take-overs).

Corporations pay tax just like people, and the sum of the tax paid by the corporation (26\%) plus you as the owner through dividends (19\%) is the same as if you were taxed individually (45\%), but there are three significant tax advantages as a corporation: \begin{enumerate}
\item \textbf{Small business tax deduction} -- reduces tax payable from 26\% to 13.5\% for up to \$500k in net \textit{income} (not revenue)
\item \textbf{Tax deferral} -- delays paying tax, allows you to earn income at a lower rate which can then be invested and earn interest. This is like borrowing from the Canadian Revenue Agency. 
\item \textbf{Capital gains exemption} -- reduces tax payable. 
\item \textbf{Qualified small business shares} -- if you sell your business you can save up to \$200k in tax, but if you did not incorporate, a good tax lawyer can hook you up to get this
\end{enumerate}

The best structure for getting investment is being a corporation. You are prohibited from selling shares to the public, and must rely on exemptions such as friends and family, or  you get an accredited investor exemption.

You then present your idea to BCIC, VANTEC, Angel Investors Forum, or whatever, then you find investors and negotiate terms. You'll want to sell the smallest slice possible for as much as possible, and investors will want the biggest slice possible for as little as possible.

At 1/3 of the company sold, you lose the rights to sell all assets or issue new shares and others. At 1/2 of the company sold, you lose the right to elect the board of directors, losing control.

\noindent You should structure your business as a corporation if: \begin{enumerate}
\item When you get an employee (usually)
\item When you are making more \$ than you spend
\item When you are dealing with big money, the loss of which you could not absorb
\item When you are doing something financially risky
\item When you have investors or partners who don't participate in the business
\item when you want to look more official than you are
\end{enumerate}

\section{Intellectual Property}

IP is increasingly important especially for engineers as companies are getting more aggressive in protecting IP. Employment agreements will likely say what happens to IP and Confidential Information.

\subsection{Patents}

Patents exist too protect the efforts of an inventor. A patent will grant you a time-limited monopoly for 20 years from the date of filing, so no one else can making, use, or sell your invention.

Patentable inventions must be \textbf{novel}, \textbf{non-obvious}\footnote{As in you cannot be a simple combination of 2 other inventions}, and \textbf{useful}. Computer software is not patentable on its own without physical implementation. Improvements and new applications of existing inventions are patentable.

It can take years to get a patent, and can cost \$5k for extremely simple objects and over \$20k for software-related inventions, for filing fees, and lawyer/patent agent fees. 

\subsubsection{Patent Infringement}

With a patent, you can sue someone for making, using, or selling your invention. Remedies for infringement are powerful and broad: \textbf{injunctions} (order to stop the infringement), \textbf{order delivery of infringing goods}, \textbf{lost profits}, \textbf{disgorgement of profits}.


\subsection{Copyrights}

Laws created to protect the efforts of artists, but this now includes software enginers, photographers, website owners, video game producers, broadcasters, and other engineers. These do not protect \textit{ideas} but rather their \textit{expressions}. Unlike patents, copyright arises automatically, you do not need to register or use \copyright.

This gives you the exclusive right to reproduce work, make recordings, perform a performance, broadcast a work, etc. A work requires some exercise in skill to create something original. 

A copyright lasts for the life of the author plus 50 years (70 years in the US). After this, the work becomes public domain. 

If you create a work while employed and at work, your employer owns it unless you agree otherwise.

\subsubsection{Infringement}

When a substantial part of a  work is copied, you can sue for infringement. Remedies are similar to patents (injunction, accounting for profits, etc).

There are also statutory remedies -- set amounts for infringement \$100-\$5k for all infringements that are non-commercial.

Fair dealing allows some copying.

\subsection{Trade-Marks}

These protect brands from being imitated and protect consumers from purchasing knock-offs. A trade-mark signals a distinguished good or service of one supplier from another. 

You can register your trade-mark for \textbf{free}, but it requires \textbf{distinctiveness}. Even without trademark, you are protected in areas where you have established "goodwill" and people know your brand.

If someone puts your mark  on their wares, you can sue for infringement with similar remedies to patents and copyright.



%--------------------------------%







\newpage
\bibliography{references}
\bibliographystyle{apacite} 
\pagebreak


% -------- APPENDIX -------- %
\appendix
\onehalfspacing
\section*{Appendix}
\addcontentsline{toc}{section}{Appendix}
\renewcommand{\thesubsection}{\Alph{subsection}}


\end{document}
